\documentclass{article}

% Language setting
% Replace `english' with e.g. `spanish' to change the document language
\usepackage[english]{babel}

% Set page size and margins
% Replace `letterpaper' with`a4paper' for UK/EU standard size
\usepackage[letterpaper,top=2cm,bottom=2cm,left=3cm,right=3cm,marginparwidth=1.75cm]{geometry}

% Useful packages
\usepackage{amsmath}
\usepackage{graphicx}
\usepackage[colorlinks=true, allcolors=blue]{hyperref}

\title{Machine Learning \\ Exercise 3: Usage of different regression algorithms in python}
\author{Prof. Dr. Thomas Kopinski}

\begin{document}
\maketitle

\begin{abstract}
The goal of this exercise is to use python and scikit-learn to perform logistic and polynomial regression on different datasets.
\end{abstract}

\section*{Task 1: Getting familiar with the regression algorithms}

\begin{itemize}
    \item Additional information about implementing polynomial regression in python can again be found \href{https://github.com/DataScienceLabFHSWF/machine-learning-book/blob/main/notebooks/ch09/ch09.ipynb}{here}.
\end{itemize}

\section*{Task 2: Logistic Regression}

\begin{itemize}
    \item Follow the tutorial \href{https://docs.anaconda.com/anaconda/install/index.html}{here} for an installation to Anaconda
    \item \href{https://docs.anaconda.com/anaconda/user-guide/getting-started/}{Here} you can get started with working with Anaconda and Launch your own Jupyter Notebooks
    \item Make sure to follow the whole guide and also work with the Anaconda prompt
    \item Generate a 'hello, world' output, either in a Notebook or in the prompt
\end{itemize}

\section*{Task 3: Polynomial Regression}

\begin{itemize}
    \item Combine both working environments to work with them together
    \item Inspect the repo - there are files in the \verb+code/ntbks+  folder
    \item Read, run and inspect the notebooks and files in the \verb+01_intro+ folder

\end{itemize}

%\bibliographystyle{alpha}
%\bibliography{sample}

\end{document}
