\documentclass{article}

% Language setting
% Replace `english' with e.g. `spanish' to change the document language
\usepackage[english]{babel}

% Set page size and margins
% Replace `letterpaper' with`a4paper' for UK/EU standard size
\usepackage[letterpaper,top=2cm,bottom=2cm,left=3cm,right=3cm,marginparwidth=1.75cm]{geometry}

% Useful packages
\usepackage{amsmath}
\usepackage{graphicx}
\usepackage[colorlinks=true, allcolors=blue]{hyperref}

\title{Machine Learning \\ Exercise 5: Principal Component Analysis}
\author{Prof. Dr. Thomas Kopinski}

\begin{document}
\maketitle

\begin{abstract}
In this exercise you will learn how to use Principal Component Analysis (PCA) to extract meaningful information from a large dataset by reducing the dimension of the provided data.
\end{abstract}

\section*{Task 1: Understand the algorithm}

\begin{itemize}
    \item Further information about PCA in general as well as examples of python implementations can be found \href{https://github.com/DataScienceLabFHSWF/machine-learning-book/blob/main/notebooks/ch05/ch05.ipynb}{here}.
    \item Make sure you understand what PCA does and how it works.
\end{itemize}

\section*{Task 2: }

\begin{itemize}
    \item Follow the tutorial \href{https://docs.anaconda.com/anaconda/install/index.html}{here} for an installation to Anaconda
    \item \href{https://docs.anaconda.com/anaconda/user-guide/getting-started/}{Here} you can get started with working with Anaconda and Launch your own Jupyter Notebooks
    \item Make sure to follow the whole guide and also work with the Anaconda prompt
    \item Generate a 'hello, world' output, either in a Notebook or in the prompt
\end{itemize}

\section*{Task 3: }

\begin{itemize}
    \item Combine both working environments to work with them together
    \item Inspect the repo - there are files in the \verb+code/ntbks+  folder
    \item Read, run and inspect the notebooks and files in the \verb+01_intro+ folder

\end{itemize}

%\bibliographystyle{alpha}
%\bibliography{sample}

\end{document}
