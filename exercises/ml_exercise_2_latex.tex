\documentclass{article}

% Language setting
% Replace `english' with e.g. `spanish' to change the document language
\usepackage[english]{babel}

% Set page size and margins
% Replace `letterpaper' with`a4paper' for UK/EU standard size
\usepackage[letterpaper,top=2cm,bottom=2cm,left=3cm,right=3cm,marginparwidth=1.75cm]{geometry}

% Useful packages
\usepackage{amsmath}
\usepackage{graphicx}
\usepackage[colorlinks=true, allcolors=blue]{hyperref}

\title{Machine Learning \\ Exercise 2: Manual implementation of linear regression in python}
\author{Prof. Dr. Thomas Kopinski}

\begin{document}
\maketitle

\begin{abstract}
This week's tasks focus on deepening your knowledge about linear regression by manually implementing the underlying algorithm in python. 
\end{abstract}

\section*{Task 1: Understand the algorithm}

\begin{itemize}
    \item Additional information about implementing linear regression in python can again be found \href{https://github.com/DataScienceLabFHSWF/machine-learning-book/blob/main/notebooks/ch09/ch09.ipynb}{here}.
    \item Make sure you understand how the algorithm works before you start implementing any python code.
    \item Try to structure your code in a clear way, e.g. by using functions.
\end{itemize}

%\bibliographystyle{alpha}
%\bibliography{sample}

\end{document}

