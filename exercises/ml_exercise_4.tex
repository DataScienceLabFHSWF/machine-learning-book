\documentclass{article}

% Language setting
% Replace `english' with e.g. `spanish' to change the document language
\usepackage[english]{babel}

% Set page size and margins
% Replace `letterpaper' with`a4paper' for UK/EU standard size
\usepackage[letterpaper,top=2cm,bottom=2cm,left=3cm,right=3cm,marginparwidth=1.75cm]{geometry}

% Useful packages
\usepackage{amsmath}
\usepackage{graphicx}
\usepackage[colorlinks=true, allcolors=blue]{hyperref}

\title{Machine Learning \\ Exercise 4: Bayes Theorem and maximum likelihood estimation}
\author{Prof. Dr. Thomas Kopinski}

\begin{document}
\maketitle

\begin{abstract}
This week you will have to implement different tasks in python to get more familiar with Bayes theorem. Additionally this exercise focuses on broadening your knowledge about solving regression problems in python by using maximum likelihood estimation (MLE). 
\end{abstract}

\section*{Task 1:}

\begin{itemize}
   % \item Additional information about implementing linear regression in python can again be found \href{https://github.com/DataScienceLabFHSWF/machine-learning-book/blob/main/notebooks/ch09/ch09.ipynb}{here}.
    \item No addtional information in the forked repo?!
\end{itemize}

\section*{Task 2: Bayes}

\begin{itemize}
    \item Follow the tutorial \href{https://docs.anaconda.com/anaconda/install/index.html}{here} for an installation to Anaconda
    \item \href{https://docs.anaconda.com/anaconda/user-guide/getting-started/}{Here} you can get started with working with Anaconda and Launch your own Jupyter Notebooks
    \item Make sure to follow the whole guide and also work with the Anaconda prompt
    \item Generate a 'hello, world' output, either in a Notebook or in the prompt
\end{itemize}

\section*{Task 3: MLE}

\begin{itemize}
    \item Combine both working environments to work with them together
    \item Inspect the repo - there are files in the \verb+code/ntbks+  folder
    \item Read, run and inspect the notebooks and files in the \verb+01_intro+ folder

\end{itemize}

%\bibliographystyle{alpha}
%\bibliography{sample}

\end{document}
