\documentclass{article}

% Language setting
% Replace `english' with e.g. `spanish' to change the document language
\usepackage[english]{babel}

% Set page size and margins
% Replace `letterpaper' with`a4paper' for UK/EU standard size
\usepackage[letterpaper,top=2cm,bottom=2cm,left=3cm,right=3cm,marginparwidth=1.75cm]{geometry}

% Useful packages
\usepackage{amsmath}
\usepackage{graphicx}
\usepackage[colorlinks=true, allcolors=blue]{hyperref}

\title{Machine Learning \\ Exercise 8: Using random forest classifiers in python}
\author{Prof. Dr. Thomas Kopinski}

\begin{document}
\maketitle

\begin{abstract}
In this exercise you will learn how to implement and use a random forest classifier in python by utilizing the scikit-learn module.
\end{abstract}

\section*{Task 1: }

\begin{itemize}
    \item Additional information and examples about implementing and using decision trees and random forests for regression can be found at the bottom of  \href{https://github.com/DataScienceLabFHSWF/machine-learning-book/blob/main/notebooks/ch09/ch09.ipynb}{this} jupyter notebook.
\end{itemize}

\section*{Task 2: }

\begin{itemize}
    
\end{itemize}

\section*{Task 3: }

\begin{itemize}
   

\end{itemize}

%\bibliographystyle{alpha}
%\bibliography{sample}

\end{document}
